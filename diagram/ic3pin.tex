% We use tikz to construct a shape corresponding to a 3 pin IC
%
% Source: http://www.elfsoft2000.com/projects/multipole.pdf


\newlength{\side}               % We define a new length that corresponds to the side of the IC we are constructing
\setlength{\side}{1.25cm}

\pgfdeclareshape{ic3pin}
{
   \anchor{center}{\pgfpointorigin}      % All circuitikz objects must have a 'center' and a 'text' anchor. Within the object/node (0, 0) is at the center and everything is drawn around this
   \anchor{text}                        % Used to center the text inside the node
      {\pgfpoint{-.5\wd\pgfnodeparttextbox}{-.5\ht\pgfnodeparttextbox}}

   % Construct custom anchors
   \savedanchor\icpina{\pgfpoint{0}{0.5\side}}        % Create an anchor with internal name \icpina for pin 1 at the specified point which is on the top of the IC at its center hence the coordinates (0, 0.5\side)
   \anchor{pin1}{\icpina}                               % We create an anchor with external name pin1 from the saved anchor created above

   \savedanchor\icpinb{\pgfpoint{0}{-0.5\side}}
   \anchor{pin2}{\icpinb}

   \savedanchor\icpinc{\pgfpoint{0.5\side}{0}}
   \anchor{pin3}{\icpinc}

   % Now we draw the actual node/object
   \foregroundpath              % Border and pin numbers are drawn here
   {
      \pgfsetlinewidth{0.03cm}

      \pgfpathrectanglecorners{\pgfpoint{-0.5\side}{-0.5\side}}{\pgfpoint{0.5\side}{0.5\side}}
      \pgfusepath{draw}     % Draw the rectangle defined above

      % Insert pin labels at the specified location with the text aligned with this point as specified (top, bottom, right, etc.)
      \pgftext[top, at={\pgfpoint{0}{0.425\side}}]{\tiny 1}
      \pgftext[bottom, at={\pgfpoint{0}{-0.425\side}}]{\tiny 2}
      \pgftext[right, at={\pgfpoint{0.425\side}{0}}]{\tiny 3}
   }
}
