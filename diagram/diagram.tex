% Tutorial Source: http://www.latex-tutorial.com/tutorials/advanced/lesson-12/

\documentclass{article}

\usepackage{amsmath}
\usepackage{tikz}
\usepackage[siunitx]{circuitikz}

% We define a new command which uses sub-stack from the amsmath package to create two line labels 
\newcommand\twoline[2]{$\substack{#1\\#2}$}         % Note that \substack must be containined within the math env $$. The advantage is that we don't have to specify $ around the arguments we provide to \twoline


\begin{document}

   \begin{center}

      \begin{figure}[h!]

         \begin{circuitikz}

            \tikzstyle{every node}=[font=\tiny]         % We reduce the font size of all text rendered in the circuit diagram

            % We reduce the size of all components in the diagram
            \ctikzset{bipoles/length=0.6cm}
            % We increase the size of the battery for aesthetic reasons
            \tikzset{battery/.append style={bipoles/length=1cm}}



            % Begin specifying the circuit
            \draw (1, 3) to[short, o-] (3, 3);

            \draw (1, 0) to[short, o-] (3, 0);

            \draw (2, 0) node[ground] {};

            \draw (0.25, 3) node[] {$+\SI{12}{\volt}$ DC};

            \draw (0.25, 0) node[] {DC GND};

            \draw (3, 0)
            to[battery, l=\twoline{\SI{12}{\volt}}{\SI{4.5}{\ampere\hour}}] (3, 2)
            to[R, l=\twoline{\SI{68}{\ohm}}{\SI{0.5}{\watt}}] (3, 3)
            to[short] (4.5, 3)
            ;

            \draw (3, 2)
            to[short] (4.5, 2)
            to[diode, l_={1N4007}] (4.5, 3)      % we use 'l_' to specify that the label should appear on the right of the diode
            ;

            \draw (4.5, 2)
            to[pR, l_={$\SI{10}{k\ohm}$}] (4.5,0)
            to[short] (3, 0)
            ;

         \end{circuitikz}


      \end{figure}

   \end{center}

\end{document}
