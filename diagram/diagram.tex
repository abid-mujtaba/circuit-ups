% Tutorial Source: http://www.latex-tutorial.com/tutorials/advanced/lesson-12/

\documentclass{article}

\usepackage{tikz}
\usepackage[siunitx]{circuitikz}


% We add an 'lx' command to circuittikz which allows for two-line labels.
%
% Source: http://tex.stackexchange.com/a/65792/50416

\makeatletter
\ctikzset{lx/.code args={#1 and #2}{ 
\pgfkeys{/tikz/circuitikz/bipole/label/name=\parbox{1cm}{\centering #1  \\ #2}}
   \ctikzsetvalof{bipole/label/unit}{}
   \ifpgf@circ@siunitx 
      \pgf@circ@handleSI{#2}
      \ifpgf@circ@siunitx@res 
            \edef\pgf@temp{\pgf@circ@handleSI@val}
            \pgfkeyslet{/tikz/circuitikz/bipole/label/name}{\pgf@temp}
            \edef\pgf@temp{\pgf@circ@handleSI@unit}
            \pgfkeyslet{/tikz/circuitikz/bipole/label/unit}{\pgf@temp}
      \else
      \fi
   \else
   \fi
}}

\ctikzset{lx^/.style args={#1 and #2}{ 
   lx=#2 and #1,
   \circuitikzbasekey/bipole/label/position=90 } 
}

\ctikzset{lx_/.style args={#1 and #2}{ 
   lx=#1 and #2,
   \circuitikzbasekey/bipole/label/position=-90 } 
}
\makeatother
    % Include auxiliary file which defines two-line labels


\begin{document}

   \begin{center}

      \begin{figure}[h!]

         \begin{circuitikz}

            \tikzstyle{every node}=[font=\tiny]         % We reduce the font size of all text rendered in the circuit diagram

            % We reduce the size of all components in the diagram
            \ctikzset{bipoles/length=0.6cm}
            % We increase the size of the battery for aesthetic reasons
            \tikzset{battery/.append style={bipoles/length=1cm}}


            % Begin specifying the circuit
            \draw (1, 3) to[short, o-] (3, 3);

            \draw (1, 0) to[short, o-] (3, 0);

            \draw (2, 0) node[ground] {};

            \draw (0.25, 3) node[] {$+\SI{12}{\volt}$ DC};

            \draw (0.25, 0) node[] {DC GND};

            \draw (3, 0)
            to[battery, lx={$\SI{12}{\volt}$ and $\SI{4.5}{\ampere\hour}$}] (3, 2)
            to[R, lx={$\SI{68}{\ohm}$ and $\SI{0.5}{\watt}$}] (3, 3)
            to[short] (5, 3)
            ;

            \draw (3, 2)
            to[short] (5, 2)
            to[diode, l_={1N4007}] (5, 3)      % we use 'l_' to specify that the label should appear on the right of the diode
            ;

            \draw (5, 2)
            to[pR, l_={$\SI{10}{k\ohm}$}] (5,0)
            to[short] (3, 0)
            ;

         \end{circuitikz}


      \end{figure}

   \end{center}

\end{document}
